\documentclass[18pt]{beamer}
\usepackage{templates/beamerthemekit}
\usepackage[export]{adjustbox}
\usepackage{tikz}
\usepackage{url}
\usepackage{amsmath}
\usepackage{marvosym} % \MVRIGHTarrow
\usepackage{stmaryrd} % \shortrightarrow
\usepackage{textcomp} % \textrightarrow
\usepackage{svg}

\title[Project discussion]{Towards Bringing Together Numerical Methods for Partial Differential Equation and Deep Neural Networks}
\subtitle{Project presentation, Supervisor - Markus Hoffmann}
\author{Stanislav Arnaudov}
\institute{Chair for Computer Architecture and Parallel Processing}
\date{4. Juni, 2020}
\selectlanguage{english}

\newcommand{\semitransp}[2][35]{\color{fg!#1}#2 \color{fg}}

\begin{document}

\begin{frame}
 \titlepage
\end{frame}

\section{Introduction}

\begin{frame}
  \frametitle{Introduction}
  \textbf{Basic idea:} Apply machine learning and image processing in computational fluid dynamics context

  \begin{columns}[t]
    \begin{column}{0.5\textwidth}

      \begin{itemize}
      \item Machine learning through deep neural networks
      \item DNNs for image generation  
      \end{itemize}
      
    \end{column}
    \begin{column}{0.5\textwidth}

      \begin{itemize}
      \item Simulation of fluids through mathematical models
      \item Differential equations
      \item Specialized solvers for the equations
      \end{itemize}
      
    \end{column}
  \end{columns}
  
\end{frame}

\begin{frame}
  \frametitle{Project}
  \begin{itemize}
  \item \textbf{What} we aimed to do?
  \item \textbf{Why} is our research being done?
  \item \textbf{How} were our goals achieved?
  \end{itemize}
\end{frame}

\section{What}

\begin{frame}
  \frametitle{What}
  \begin{itemize}
  \item Predcit the flow around an object in a channel.
  \end{itemize}
\end{frame}

\begin{frame}
  \frametitle{What}
  \begin{itemize}
  \item Predcit the flow around an object in a channel.
  \end{itemize}


    \begin{columns}[t]
    \begin{column}{0.5\textwidth}

      \begin{itemize}
      \item Inflow speed of the fluid
      \item Viscosity and density of the fluid
      \end{itemize}
      
    \end{column}
    \begin{column}{0.5\textwidth}

      \begin{itemize}
      \item Velocity field (x\&y directions)
      \item Pressure field
      \end{itemize}
      
    \end{column}
  \end{columns}

\end{frame}

\begin{frame}
  \frametitle{What}
  %Diagram of the network evaluation goes here
\end{frame}

\begin{frame}
  \frametitle{What}
  %Images of solutions go here go here
\end{frame}

\section{Why}

\begin{frame}
  \frametitle{Why}

  \begin{columns}[t]
    \begin{column}{0.5\textwidth}
      
    \end{column}
    \begin{column}{0.5\textwidth}
      \begin{itemize}
      \item lots of parameters to tweak
      \item complicated workflow
      \item computationally costly
      \end{itemize}
    \end{column}
  \end{columns}  
\end{frame}

\begin{frame}
  \frametitle{Why}

  \begin{columns}[t]
    \begin{column}{0.5\textwidth}
      
    \end{column}
    \begin{column}{0.5\textwidth}
      \begin{itemize}
      \item DNNs are well established in image processing tasks
      \item everything is learned during training
      \item faster image generation
      \end{itemize}
    \end{column}
  \end{columns}  
\end{frame}


\section{How}

\begin{frame}
  \frametitle{How}
  \begin{itemize}
  \item Machine Learning project
    \begin{itemize}
    \item Data used for training
    \item Models architecture and training approach
    \item Models evaluation
    \end{itemize}
  \item Not a single holistic model
    \begin{itemize}
    \item constant model
    \item inflow speed model
    \item viscosity-density model
    \end{itemize}
    
  \end{itemize}
\end{frame}

\section{Data Generation}

\begin{frame}
  \frametitle{Data Generation}
  \begin{itemize}
  \item Process
    \begin{itemize}
    \item Large amount of simulations
    \item Solving --- HiFlow3 as numerical solver
    \item Rendering --- ParaView as visualization toolkit
    \end{itemize}

  \item Separate data sets for each model
    \begin{itemize}
    \item Single simulation for the constant model
    \end{itemize}
  \end{itemize}
\end{frame}

\begin{frame}
  \frametitle{Data Generation}
  \begin{itemize}
  \item Parameters for the simulations
    \begin{itemize}
    \item Not chosen at random
    \item \textit{Reynolds number} in [90, 450]
    \item Non-trivial simulations
    \end{itemize}
  \end{itemize}
\end{frame}

\begin{frame}
  \frametitle{Data Generation}
  % images come here
\end{frame}

\begin{frame}
  \frametitle{Data Generation}

  \begin{itemize}
  \item Test train splits
    \begin{itemize}
    \item 80\slash 20 split for all data sets
    \item no common Reynolds numbers between the simulation in the train- and test-sets
    \end{itemize}
  \end{itemize}
\end{frame}

\begin{frame}
  \frametitle{Networks Architectures}
  \begin{itemize}
  \item Real numbers handling
  \end{itemize}
\end{frame}

\begin{frame}
  \frametitle{Networks Architectures}
  \begin{itemize}
  \item Approach based on Pix2Pix
    \begin{itemize}
    \item General image-to-image translation framework
    \item impressive results
    \end{itemize}
  \end{itemize}
  % images goes here
\end{frame}

\begin{frame}
  \frametitle{Networks Architectures}
  \begin{itemize}
  \item Approach based on Pix2Pix
  \item Conditional GAN
  \end{itemize}
  % image of cGAN goes here
\end{frame}


\begin{frame}
  \frametitle{Networks Architectures}
  \begin{itemize}
  \item Approach based on Pix2Pix
  \item Conditional GAN
  \item Generator Architecture
    \begin{itemize}
    \item UNet
    \end{itemize}
  \end{itemize}
  % image of Unet goes here
\end{frame}


\begin{frame}
  \frametitle{Networks Architectures}
  \begin{itemize}
  \item Approach based on Pix2Pix
  \item Conditional GAN
  \item Generator Architecture
  \item Discriminator Architecture
    \begin{itemize}
    \item PatchGAN
    \end{itemize}
  \end{itemize}
  % image of cGAN goes here
\end{frame}

\section{Conclusion}

\begin{frame}
  \frametitle{}
  \begin{center}
    \huge{Thank you for your attention.}
  \end{center}
\end{frame}

\begin{frame}
  \frametitle{}
  \begin{center}
    \huge{Questions?}
  \end{center}
\end{frame}

\end{document}

%%% Local Variables:
%%% mode: latex
%%% TeX-master: t
%%% End:
