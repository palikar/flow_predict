\documentclass[18pt]{beamer}
\usepackage{templates/beamerthemekit}
\usepackage[export]{adjustbox}
\usepackage{tikz}
\usepackage{url}
\usepackage{amsmath}
\usepackage{marvosym} % \MVRIGHTarrow
\usepackage{stmaryrd} % \shortrightarrow
\usepackage{textcomp} % \textrightarrow
\usepackage{svg}

\title[Progress Update]{Towards Bringing Together Numerical Methods for Partial Differential Equation and Deep Neural Networks}
\subtitle{Progress Update, Supervisor - Markus Hoffmann}
\author{Stanislav Arnaudov}
\institute{Chair for Computer Architecture and Parallel Processing}
\date{September 26, 2019}
\selectlanguage{english}

% \usepackage[style=verbose,backend=bibtex]{biblatex}
% \bibliography{bib}
% \bibliographystyle{plain}

% \newcommand{\semitransp}[2][35]{\color{fg!#1}#2 \color{fg}}

\begin{document}

\begin{frame}
 \titlepage
\end{frame}



\begin{frame}
 \titlepage
\end{frame}


\section{Description}

\begin{frame}[t]
  \frametitle{Project description}
  \begin{center}
    \large{\textbf{Basic idea:} Perform numerical simulation with ML-models}
  \end{center}
\end{frame}


\begin{frame}[t]
  \frametitle{Project description}
  \begin{center}
    \large{\textbf{Basic idea:} Perform numerical simulation with ML-models}
  \end{center}
  
  \begin{itemize}
  \item Concrete problem: Flow around an object according to the Navier–Stokes equations.
  \end{itemize}
  % simulation setup here
\end{frame}

\begin{frame}[t]
  \frametitle{Project description}
  \begin{center}
    \large{\textbf{Basic idea:} Perform numerical simulation with ML-models}
  \end{center}
  \begin{itemize}
  \item Solutions of the simulation can be represented as images.
  \end{itemize}
  % simulation image here
\end{frame}


\begin{frame}[t]
  \frametitle{Project description}
  \begin{center}
    \large{\textbf{Basic idea:} Perform numerical simulation with ML-models}
  \end{center}
  
  \begin{itemize}
  \item Or ML-model primarily use images as input and output.
  \end{itemize}
  % system overview here
\end{frame}

\begin{frame}[t]
  \frametitle{Project description}
  \large{Several cases to investigate}
  \begin{itemize}
  \item Constant model
  \item Fluid speed model
  \item Fluid viscosity and density model
  \item Object in space model
  \end{itemize}
  
  
  
  % system overview here
\end{frame}

\section{Data}
\begin{frame}[t]
  \frametitle{Data Generation}
  \begin{itemize}
  \item Use of numerical solver for real simulation data generation.
  \item The simulation has several adjustable parameters
    \begin{itemize}
    \item inflow speed
    \item fluid viscosity
    \item fluid density
    \end{itemize}
  \item Reynolds Number in the range of [90, 350]
  \end{itemize}  
\end{frame}

\begin{frame}[t]
  \frametitle{Data Generation}
  \begin{itemize}
  \item Use of numerical solver for real simulation data generation.
  \item The simulation has several adjustable parameters
  \item Reynolds Number in the range of [90, 350]
  \end{itemize}
\end{frame}


\begin{frame}[t]
  \frametitle{Data Generation}
  \begin{itemize}
  \item Use of numerical solver for real simulation data generation.
  \item The simulation has several adjustable parameters
  \item Reynolds Number in the range of [90, 350]
  \end{itemize}
  % karman vortex street here
\end{frame}

\begin{frame}[t]
  \frametitle{Data Generation}
  \begin{itemize}
  \item Use of numerical solver for real simulation data generation.
  \item The simulation has several adjustable parameters
  \item Reynolds Number in the range of [90, 350]
  \item Choosing appropriate color space
    \begin{itemize}
    \item RGB
    \item Grayscale
    \end{itemize}
  \end{itemize}
\end{frame}

\section{Models}
\begin{frame}[t]
  \frametitle{Models}
  \begin{itemize}
  \item Two types of architectures based on our preliminary research:
  \end{itemize}
\end{frame}


\begin{frame}[t]
  \frametitle{Models}
  \begin{itemize}
  \item Two types of architectures based on our preliminary research:
    \begin{itemize}
    \item ResNet 
    \end{itemize}
    % image of res net here
  \end{itemize}
\end{frame}


\begin{frame}[t]
  \frametitle{Models}
  \begin{itemize}
  \item Two types of architectures based on our preliminary research:
    \begin{itemize}
    \item UNet
    \end{itemize}
    % image of unet here
  \end{itemize}
\end{frame}


\begin{frame}[t]
  \frametitle{Models}
  \begin{itemize}
  \item Two types of architectures based on our preliminary research:
    \begin{itemize}
    \item UNet turned out to perform better.
    \end{itemize}
  \end{itemize}
\end{frame}


\begin{frame}[t]
  \frametitle{Models}
  \begin{itemize}
  \item Two types of architectures based on our preliminary research:
  \item Data being used by the network.
  \end{itemize}
\end{frame}

\begin{frame}[t]
  \frametitle{Models}
  \begin{itemize}
  \item Two types of architectures based on our preliminary research:
  \item Data being used by the network.
    \begin{itemize}
    \item Usage of pressure field
    \end{itemize}
    % pressure optional image here
  \end{itemize}
\end{frame}

\begin{frame}[t]
  \frametitle{Models}
  \begin{itemize}
  \item Two types of architectures based on our preliminary research:
  \item Data being used by the network.
    \begin{itemize}
    \item Processing of real values
    \end{itemize}
    % 3.14? image here
  \end{itemize}
\end{frame}


\begin{frame}[t]
  \frametitle{Models}
  \begin{itemize}
  \item Two types of architectures based on our preliminary research:
  \item Data being used by the network.
    \begin{itemize}
    \item Usage of pressure field $\rightarrow$ the pressure field turned out to be useful
    \item Processing of real values $\rightarrow$ extra image channel filled with the value
    \end{itemize}
    % image 3.14 with grid here
  \end{itemize}
\end{frame}

\section{Results}
\begin{frame}
  \frametitle{Evaluating the results}  
\end{frame}

\begin{frame}
  \frametitle{Evaluation cases}
\end{frame}

\begin{frame}
  \frametitle{}
  \begin{center}
    \huge{Thank you for your attention.}
  \end{center}
\end{frame}

\begin{frame}
  \frametitle{}
  \begin{center}
    \huge{Questions?}
  \end{center}
\end{frame}

\end{document}

%%% Local Variables:
%%% mode: latex
%%% TeX-master: t
%%% End:
